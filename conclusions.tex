In this paper we have analyzed the implementation of a coarse-grain 
parallelization of the DNN training process. The approach has to 
significant properties. On one side, the coarse-grain parallelization 
is network-agnostic in the sense that it does not rely on any 
specialized and higly tuned library for the specificities of the 
netowrk layers. Besides, it does not require the programming efforts 
related to recoding the layer implementation to adapt to such libraries. 
Second, it is convergence-invariant as it does not change any of the 
training parameters, thus ensuring consistency with the sequential 
convergence rate.  The implementation has 
been done in Caffe, a general and research oriented DNN framework. 
The coarse-grain parallelization follows a directive-based style, 
using OpenMP directives. We have identified its main 
limiting performance factors as well as the main advantages and 
disadvantages in front of a fine-grain parallelization.
In particular, the implementationn of fine-grain strategies requires
the recoding of all network transformations in order to port them 
to the GPU device. For two state-of-the-art datasets (MNIST and CIFAR-10), 
the observed performance levels are similar for both startegies. 
Only when using specialized libraries available for specific transformations 
like convolutional and pooling operations the fine-grain startegy makes a 
difference in terms of performance and justifies all the programming efforts 
associated to this approach.
For future work the challenges are clear: switch to a parallel memory allocation and enabling the multi-GPU execution. Both features will translate in great performance improvements for the coarse-grain approach.
