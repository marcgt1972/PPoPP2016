Recently, deep neural networks (DNN) have achieved extraordinary 
results in domains like computer vision and speech recognition \cite{ILSVRC15, Hannun2014}. 
One key and essential element for this success
has been the introduction of high performance computing (HPC)
techniques within one critical step in the network deployment: the
network training. 
Neural networks require a training stage which typically has
very high computational demands. 
Currently, the optimization of this stage
has become the main bottleneck for a reasonable trade-off between the
final network accuracy and the actual length of the
training process \cite{JeffDean2012, Coates2013} 
% [Downpour SGD and Sandblaster L-BFGS]. 

%In most recent and successful proposals, the training stage
%takes about \todo{XXX hours} for a network with state-of-the-art 
%accuracy [REFs]. Clearly, optimizing and accelerating this process are
%key priorities in the DNN domain.

During the training process, a gradient descent algorithm minimizes a pre-defined cost function. 
Most computations
along this process correspond to basic linear algebra operations. In
general, an optimal training algorithm is built upon a very specific 
data layout mainly composed of matrices and vectors, and an iterative
procedure that invokes specialized and highly optimized basic
linear algebra subroutines (BLAS).
All the widely used DNN frameworks (e,g.: Caffe \cite{Caffe}, Theano \cite{Theano} or Torch \cite{Torch}) follow this approach. 
When it comes to optimization, 
GPU acceleration has been the most widespread adopted strategy and they have
shown extraordinary performance levels. 
GPU's are massively
parallel architectures for very fine grain parallelism and BLAS
computations make a perfect fit for them. 
But in terms of programmability,
porting the original code to GPU kernels requires significant programming 
efforts. 
The general solution has been to use specialized 
libraries like cuBLAS and cuDNN \cite{chetlur2014cudnn} . 
Unfortunately, this approach lacks from generality.
During the research cycle to create a DNN, the optimal network topology,
parameters and data layout are unknown. 
So, the specialized GPU libraries are not yet useful. 
Even for the case of cuDNN, 
only DNN transformations that are very well understood 
% and no longer in a research stage 
(e.g.: convolution and pooling operations),
are supported, and has a strong bias towards computer vision applications.

An unexplored strategy is a CPU parallelization at a coarser level.
Neural networks are trained with batches of data where each batch
contains a fixed number of data samples. Each sample is used to
advance the minimization of the cost function and all samples in
the batch can be processed in parallel. This defines a much coarser
thread level parallelism but adds complexity to the
parallelization process: thread level parallelism has to be explicitly 
introduced and race conditions appear for the network coefficient 
update. 
%This has made this option not fully explored by the
%DNN community although its potential is very significant. 
The batch-level parallelism is inherent to the gradient descent 
algorithm and produces immediate acceleration irrespective of the 
nature of the network computations. It does not rely on a 
particular optimal data layout nor the availability of any specialized 
and highly optimized library for the network layers. So, its 
activation does not require 
any code porting efforts. We name this property as \textbf{network-agnostic}.
Moreover, a batch-level parallelization does not change any 
training parameter, thus it does not affect the convergence of the 
gradient descent. This is important, because parallelization strategies 
that do change the training parameters do not ensure convergence
invariance. For instance, multi-GPU systems have been explored when 
the batch and model parameters do not fit in the memory of a single GPU.
The solution has been to reduce the batch size and make it fit in memory.
But this changes the batch size and alters the convergence. In general, 
doubling the number of GPUs having halved the batch size is never a 
guarantee of observing a 2x speedup for the training 
process \cite{Hannun2014}. We name this property as \textbf{convergence-invariant}.

%would enable a multi-GPU execution
%without changing any parameter of the training algorithm. This is
%important because the convergence rate would not be affected.
%Multi-GPU systems have been explored, but only when the batch
%and model parameters do not fit in the memory of a single GPU.
%The solution has been to reduce the batch size and switch to 
%multiple GPUs. But this approach changes the batch size and thus
%alters the convergence. In general, doubling the number of GPUs
%having halved the batch size is never a guarantee of observing a
%2x speedup for the training process [Baidu deep speech].

The main contribution of this paper is the implementation and
analysis of a \textbf{network-agnostic} and \textbf{convergence-invariant} 
coarse-grain parallelization of the DNN training algorithm.
It exploits a parallelism level that is independent from the
support of specialized and optimized libraries. Therefore, the
optimization is immediately available for accelerating the DNN
training. And it is compatible with multi-GPU execution without 
altering the algorithm convergence rate. The parallelization has 
been implemented in Caffe \cite{Caffe}, a state-of-the-art DNN framework. 
The paper describes the code transformations for the parallelization 
and we also identify the limiting performance factors of the
approach and expose appropriate optimizations. We show competitive 
performance results for two state-of-the-art computer vision
datasets, MNIST \cite{MNIST,Deng2012} and CIFAR-10 \cite{CIFAR10}. 
%
In particular, in a  \todo{ 16-core Xeon E5-2667 
at 3.30GHz we observe speedups of XXX, at similar performance levels 
of those obtained by the GPU optimized Caffe version in a NVIDIA K40 GPU.}

The rest of this paper is organized as Section 2, which 
describes the main design aspects of Caffe regarding its 
parallelization. Section 3 describes the code transformations 
and optimizations for a batch-level parallelization of the 
training core of Caffe. Section 4 analyses the performance 
limiting factors of the approach and its overall performance. 
Section 5 presents the related work and finally Section 6 
concludes the paper with its main conclusions.

